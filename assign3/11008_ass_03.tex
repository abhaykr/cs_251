\documentclass{article}
\usepackage{algpseudocode}
\begin{document}
\title{CS 251}
\author{Abhay Kumar}
\maketitle
ROLL:11008\\
GROUP:G1
\begin{center}
\textbf{\huge ASSIGNMENT \#03}\\
\ \\
\ \\
\section{MULTIPLICATION OF TWO LARGE NUMBER}
\end{center}
\begin{algorithmic}
\For{$k=0 \to n-1 $}
\State $carry \gets 0$
\For{$j=0 \to n-1 $}
\State $ product \gets (array\_number1[k]*array\_number2[j]+carry)$
\State $ store\_table[j][k + j] \gets product \% 10 $
\State $carry \gets product/10$
\EndFor
\State $store\_table[j][n+j] \gets carry$
\EndFor
\For {$k=0 \to 2*n-1$}
\State $sum \gets carry$
\For {$j = 0 \to n-1$}
\State$sum \gets sum + table[j][i]$
\EndFor
\State $remainder[k] \gets  sum \% 10$
\State $carry \gets sum/10$
\EndFor
\State $remainder[2*n] \gets carry$
\end{algorithmic}
\begin{center}
\subsection {\#THEOREY\#}
 
\end{center}
 we used algorithm as we learnt in school days.
 we used here two array storing  digit of both number of size n.
 start multiplying the first element of the digit of first array to all digit of the second array and so on so far.
 after that dividing it with 10 we taking the carry of multiplication.
 by taking modulus  with 10 we store the digit that comes after multiplication. \\
 At last that we are adding all digit that comes by multipliction with adding carries that we stored.
\ \\
\ \\
\begin{center}
\textbf{\huge TIME COMPLEXITY}

 in the first half of the program we are using two loops from 0 to n-1 .
 in second half of the program we also using two loops but another from
 0 to 2n-1 and other one from 0 to$ n-1$

 so we have time complexity of $O(n^2)$ where n is the number of digit. 

\end{center}
\end{document}        